% $Date$
% $Revision$
% $Author$

%%%%%%%%%%%%%%%%%%%%%%%%%%%%%%%%%%%%%%%%%%%%%%%%%%%%%%%%%%%%%%%%%%%%%%%%%%%%%%%%%%%%%%%%%%%%%%%%%%%
%                                                                                                 %
% The mathematical style of these documents follows                                               %
%                                                                                                 %
% A. Thompson and B.N. Taylor. The NIST Guide for the Use of the International System of Units.   %
%    NIST Special Publication 881, 2008.                                                          %
%                                                                                                 %
% http://www.nist.gov/pml/pubs/sp811/index.cfm                                                    %
%                                                                                                 %
%%%%%%%%%%%%%%%%%%%%%%%%%%%%%%%%%%%%%%%%%%%%%%%%%%%%%%%%%%%%%%%%%%%%%%%%%%%%%%%%%%%%%%%%%%%%%%%%%%%

% Packages which force the use of better TeX coding
% Mostly from http://tex.stackexchange.com/q/19264
%%\RequirePackage[l2tabu, orthodox]{nag}
%%\usepackage{fixltx2e}
%\usepackage{isomath} % Disabled for the moment because it changes the syntax for bold and roman Greek math symbols
%%\usepackage[all,warning]{onlyamsmath}
%\usepackage{strict} % Commented out for now because it is uncommon. A copy of style.sty is in Manuals/LaTeX_Style_Files/.

\usepackage{times,mathptmx}
\usepackage[pdftex]{graphicx}
\usepackage{tabularx}
\usepackage{multirow}
\usepackage{pdfsync}
\usepackage{tikz}
\usepackage{pgfplots}
%\pgfplotsset{compat=1.7}
\usepackage{tocloft}
\usepackage{color}
\usepackage{amsmath}
\definecolor{linknavy}{rgb}{0,0,0.50196}
\definecolor{linkred}{rgb}{1,0,0}
\definecolor{linkblue}{rgb}{0,0,1}
\usepackage{float}
\usepackage{caption}
\usepackage{graphpap}
\usepackage{rotating}
\usepackage{geometry}
\usepackage{relsize}
\usepackage{longtable}
\usepackage{lscape}
\usepackage{amssymb}
\usepackage{makeidx} % Create index at end of document
\usepackage[nottoc,notlof,notlot]{tocbibind} % Put the bibliography and index in the ToC
\usepackage{lastpage} % Automatic last page number reference.
\usepackage[T1]{fontenc}
\usepackage{enumerate}
\usepackage{upquote}
\usepackage{moreverb}
\usepackage{morefloats}

% Smokeview Version String
\newcommand{\smvversion}{6.6.0}

\newcommand{\nopart}{\expandafter\def\csname Parent-1\endcsname{}} % To fix table of contents in pdf.
\newcommand{\ct}{\tt\small} % eventually will be deprecated due to http://www.tex.ac.uk/cgi-bin/texfaq2html?label=2letterfontcmd
\newcommand{\textct}[1]{\texttt{\small #1}}

\usepackage{tocstyle} % Fix table of contents sections from overlapping section titles
\usetocstyle{standard}
\usepackage{siunitx}
\sisetup{
    detect-all = true,
    input-decimal-markers = {.},
    input-ignore = {,},
    inter-unit-product = \ensuremath{{}\cdot{}},
    multi-part-units = repeat,
    number-unit-product = \text{~},
    per-mode = fraction,
    separate-uncertainty = true,
}

\usepackage{listings}
\usepackage{textcomp}
\definecolor{lbcolor}{rgb}{0.96,0.96,0.96}
\lstset{
    %backgroundcolor=\color{lbcolor},
    tabsize=4,
    rulecolor=,
    language=Fortran,
        basicstyle=\footnotesize\ttfamily,
        upquote=true,
        aboveskip={\baselineskip},
        belowskip={\baselineskip},
        columns=fixed,
        extendedchars=true,
        breaklines=true,
        breakatwhitespace=true,
        frame=none,
        showtabs=false,
        showspaces=false,
        showstringspaces=false,
        identifierstyle=\ttfamily,
        keywordstyle=\color[rgb]{0,0,0},
        commentstyle=\color[rgb]{0,0,0},
        stringstyle=\color[rgb]{0,0,0},
}

\usepackage{xr-hyper}
\usepackage[pdftex,
        colorlinks=true,
        urlcolor=linkblue,     % \href{...}{...} external (URL)
        citecolor=linkred,     % citation number colors
        linkcolor=linknavy,    % \ref{...} and \pageref{...}
        pdfproducer={pdflatex},
        pagebackref,
        pdfpagemode=UseNone,
        bookmarksopen=true,
        plainpages=false,
        verbose]{hyperref}

% The Following commented code makes the ``Draft'' watermark on each page.
%\usepackage{eso-pic}
%\usepackage{type1cm}
%\makeatletter
%   \AddToShipoutPicture{
%     \setlength{\@tempdimb}{.5\paperwidth}
%     \setlength{\@tempdimc}{.5\paperheight}
%     \setlength{\unitlength}{1pt}
%     \put(\strip@pt\@tempdimb,\strip@pt\@tempdimc){
%     \makebox(0,0){\rotatebox{45}{\textcolor[gray]{0.75}{\fontsize{8cm}\selectfont{RC6}}}}}
% }
%\makeatother

\setlength{\textwidth}{6.5in}
\setlength{\textheight}{9.0in}
\setlength{\topmargin}{0.in}
\setlength{\headheight}{0.in}
\setlength{\headsep}{0.in}
\setlength{\parindent}{0.25in}
\setlength{\oddsidemargin}{0.0in}
\setlength{\evensidemargin}{0.0in}
\setlength{\leftmargini}{\parindent} % Controls the indenting of the "bullets" in a list
\setlength{\cftsecnumwidth}{0.45in}
\setlength{\cftsubsecnumwidth}{0.5in}
\setlength{\cftfignumwidth}{0.45in}
\setlength{\cfttabnumwidth}{0.45in}

\newcommand{\authortitlesigs}
{
\begin{flushright}
Kevin McGrattan \\
Simo Hostikka \\
Randall McDermott \\
Jason Floyd \\
Craig Weinschenk \\
Kristopher Overholt
\end{flushright}
}

\newcommand{\logosigs}{
\begin{minipage}[b]{6.5in}
\parbox[b]{3.5in}{
\includegraphics[width=1.3in]{../Bibliography/VTT_BLACK_L} \\
VTT Technical Research Centre of Finland}
\hfill
\parbox[b]{3in}{\flushright{\includegraphics[width=2.in]{../Bibliography/nistident_flright_vec}}}
\end{minipage}
}

\newcommand{\authorsigs}
{
\begin{flushright}
Kevin McGrattan \\
Randall McDermott \\
{\em Fire Research Division \\
Engineering Laboratory \\
Gaithersburg, Maryland, USA} \\[.1in]
Simo Hostikka \\
{\em Aalto University \\
Espoo, Finland} \\[.1in]
Jason Floyd \\
Craig Weinschenk \\
{\em Jensen Hughes \\
Baltimore, Maryland, USA}\\[.1in]
Kristopher Overholt \\
{\em Continuum Analytics \\
Austin, Texas, USA}
\end{flushright}
}

\newcommand{\titlesigs}
{
\small
\begin{flushright}
U.S. Department of Commerce \\
{\em Wilbur L. Ross, Jr., Secretary} \\
\hspace{1in} \\
National Institute of Standards and Technology \\
{\em Walter Copan, NIST Director and Undersecretary of Commerce for Standards and Technology}
\end{flushright}
}


\newcommand{\disclaimer}[1]{
\begin{minipage}[t][8in][s]{6.5in}
\fontsize{10}{12}\selectfont
\flushright{Certain commercial entities, equipment, or materials may be identified in this \\
document in order to describe an experimental procedure or concept adequately. \\
Such identification is not intended to imply recommendation or endorsement by the \\
National Institute of Standards and Technology, nor is it intended to imply that the \\
entities, materials, or equipment are necessarily the best available for the purpose.\\
}

\vspace{3in}

\large
\flushright{\bf National Institute of Standards and Technology Special Publication #1 \\
Natl.~Inst.~Stand.~Technol.~Spec.~Publ.~#1, \pageref{LastPage} pages (October 2013) \\
CODEN: NSPUE2 }

\vfill

\hspace{1in}

\end{minipage}
}



\newcommand{\gforneybio}
{
\item[Glenn Forney] is a computer scientist at the Engineering Laboratory of NIST.  He received a
bachelor of science degree in mathematics from Salisbury State College and a master of
science and a doctorate in mathematics from Clemson University.  He joined NIST
in 1986 (then the National Bureau of Standards) and has since worked on developing tools that
provide a better understanding of fire phenomena, most notably Smokeview, a software tool for visualizing
Fire Dynamics Simulator data.
}

\newcommand{\smvoverview}
{
This guide is part of a three volume set of companion documents describing Smokeview.
Volume I, the Smokeview User's Guide~\cite{Smokeview_Users_Guide}, describes  how to use Smokeview.
Volume II, the Smokeview Technical Reference Guide~\cite{Smokeview_Tech_Guide}, gives technical details of how the visualizations are performed.
Volume III, the Smokeview Verification Guide~\cite{Smokeview_Verification_Guide} presents example cases
verifying the various visualization capabilities of Smokeview.
   Details on the use and technical background of the Fire Dynamics Simulator is contained in the FDS User's~\cite{FDS_Users_Guide} and Technical reference guide~\cite{FDS_Math_Guide}
respectively.
}

\newcommand{\smvoverviewb}
{
This guide is part of a four volume set of companion documents describing Smokeview.
Volume I, the Smokeview User's Guide~\cite{Smokeview_Users_Guide}, describes  how to use Smokeview.
Volume II, the Smokeview Technical Reference Guide~\cite{Smokeview_Tech_Guide}, gives technical details of how the visualizations are performed.
Volume III, the Smokeview Verification Guide~\cite{Smokeview_Verification_Guide} presents example cases
verifying the various visualization capabilities of Smokeview.
Volume IV, the Smokeview Utilities Guide~\cite{Smokeview_Utilities_Guide}, describes how
to use various utilities assocated with Smokeview such as Smokediff and Smokezip.
   Details on the use and technical background of the Fire Dynamics Simulator is contained in the FDS User's~\cite{FDS_Users_Guide} and Technical reference guide~\cite{FDS_Math_Guide}
respectively.
}

% commands to use for "official" cover and title pages
% see smokeview verification guide to see how they are used

\newcommand{\headerA}[1]{
\begin{flushright}
\fontsize{20}{24}\selectfont
\bf{NIST Special Publication #1}
\end{flushright}
}


\newcommand{\headerB}[1]{
\begin{flushright}
\fontsize{28}{33.6}\selectfont
\bf{#1}
\end{flushright}
}

\newcommand{\headerC}[1]{
\vspace{.15in}
\begin{flushright}
\fontsize{12}{14}\selectfont
#1
\end{flushright}
}

\newcommand{\headerD}[1]{
\begin{flushright}
\fontsize{12}{14}\selectfont
http://dx.doi.org/10.6028/NIST.SP.#1
\end{flushright}
}


\newcommand{\vvec}[1]{\mathbf{#1}}
\newcommand{\dod}[2]{\frac{\partial #1}{\partial #2}}
\newcommand{\DoD}[2]{\frac{\mathrm{D} #1}{\mathrm{D} #2}}
\newcommand{\dsods}[2]{\frac{\partial^2 #1}{\partial #2^2}}
\renewcommand{\d}{\,\mathrm{d}}
\newcommand{\dx}{\delta x}
\newcommand{\dy}{\delta y}
\newcommand{\dz}{\delta z}
\newcommand{\degF}{$^\circ$F}
\newcommand{\degC}{$^\circ$C}
\newcommand{\x}{x}
\newcommand{\y}{y}
\newcommand{\z}{z}
\newcommand{\dt}{\delta t}
\newcommand{\dn}{\delta n}
\newcommand{\cH}{H}
\newcommand{\hu}{u}
\newcommand{\hv}{v}
\newcommand{\hw}{w}
\newcommand{\la}{\lambda}
\newcommand{\bO}{{\Omega}}
\newcommand{\bo}{{\mathbf{\omega}}}
\newcommand{\btau}{\mathbf{\tau}}
\newcommand{\bdelta}{{\mathbf{\delta}}}
\newcommand{\sumyw}{\sum (Y_\alpha/W_\alpha)}
\newcommand{\oW}{\overline{W}}
\newcommand{\om}{\ensuremath{\omega}}
\newcommand{\omx}{\omega_x}
\newcommand{\omy}{\omega_y}
\newcommand{\omz}{\omega_z}
\newcommand{\erf}{\hbox{erf}}
\newcommand{\erfc}{\hbox{erfc}}
\newcommand{\bF}{{\mathbf{F}}}
\newcommand{\bG}{{\mathbf{G}}}
\newcommand{\bof}{{\mathbf{f}}}
\newcommand{\bq}{{\mathbf{q}}}
\newcommand{\br}{{\mathbf{r}}}
\newcommand{\bu}{{\mathbf{u}}}
\newcommand{\bx}{{\mathbf{x}}}
\newcommand{\bk}{{\mathbf{k}}}
\newcommand{\bv}{{\mathbf{v}}}
\newcommand{\bg}{{\mathbf{g}}}
\newcommand{\bn}{{\mathbf{n}}}
\newcommand{\bS}{{\mathbf{S}}}
\newcommand{\bW}{\overline{W}}
\newcommand{\dS}{d{\mathbf{S}}}
\newcommand{\bs}{{\mathbf{s}}}
\newcommand{\bI}{{\mathbf{I}}}
\newcommand{\hp}{H}
\newcommand{\trho}{\tilde{\rho}}
\newcommand{\dph}{{\delta\phi}}
\newcommand{\dth}{{\delta\theta}}
\newcommand{\tp}{\tilde{p}}
\newcommand{\bp}{\overline{p}}
\newcommand{\dQ}{\dot{Q}}
\newcommand{\dq}{\dot{q}}
\newcommand{\dbq}{\dot{\mathbf{q}}}
\newcommand{\dm}{\dot{m}}
\newcommand{\ha}{\frac{1}{2}}
\newcommand{\ft}{\frac{4}{3}}
\newcommand{\ot}{\frac{1}{3}}
\newcommand{\fofi}{\frac{4}{5}}
\newcommand{\of}{\frac{1}{4}}
\newcommand{\twth}{\frac{2}{3}}
\newcommand{\R}{R}
\newcommand{\be}{\begin{equation}}
\newcommand{\ee}{\end{equation}}
\newcommand{\RE}{\hbox{Re}}
\newcommand{\LE}{\hbox{Le}}
\newcommand{\PR}{\hbox{Pr}}
\newcommand{\PE}{\hbox{Pe}}
\newcommand{\NU}{\hbox{Nu}}
\newcommand{\SC}{\hbox{Sc}}
\newcommand{\SH}{\hbox{Sh}}
\newcommand{\WE}{\hbox{We}}
\newcommand{\COTWO}{\text{\tiny \hbox{CO}$_2$}}
\newcommand{\HTWOO}{\text{\tiny \hbox{H}$_2$\hbox{O}}}
\newcommand{\OTWO}{\text{\tiny \hbox{O}$_2$}}
\newcommand{\NTWO}{\text{\tiny \hbox{N}$_2$}}
\newcommand{\CO}{\text{\tiny \hbox{CO}}}
\newcommand{\F}{\text{\tiny \hbox{F}}}
\newcommand{\C}{\text{\tiny \hbox{C}}}
\newcommand{\Hy}{\text{\tiny \hbox{H}}}
\newcommand{\So}{\text{\tiny \hbox{S}}}
\newcommand{\M}{\text{\tiny \hbox{M}}}
\newcommand{\xx}{\text{\tiny \hbox{x}}}
\newcommand{\yy}{\text{\tiny \hbox{y}}}
\newcommand{\zz}{\text{\tiny \hbox{z}}}
\newcommand{\smvlines}{135~000}

\newcommand{\calH}{\mathcal{H}}
\newcommand{\calR}{\mathcal{R}}

\newcommand{\dif}{\mathrm{d}}
\newcommand{\Div}{\nabla\cdot}
\newcommand{\D}{\mbox{D}}
\newcommand{\mhalf}{\mbox{$\frac{1}{2}$}}
\newcommand{\thalf}{\mbox{\tiny $\frac{1}{2}$}}
\newcommand{\tripleprime}{{\prime\prime\prime}}
\newcommand{\ppp}{{\prime\prime\prime}}
\newcommand{\pp}{{\prime\prime}}

\newcommand{\superscript}[1]{\ensuremath{^{\textrm{\tiny #1}}}}
\newcommand{\subscript}[1]{\ensuremath{_{\textrm{\tiny #1}}}}

\newcommand{\rb}[1]{\raisebox{1.5ex}[0pt]{#1}}

\newcommand{\Ra}{$\Rightarrow$}
\newcommand{\hhref}[1]{\href{#1}{{\tt #1}}}
\newcommand{\fdsinput}[1]{{\scriptsize\verbatiminput{../../Verification/Visualization/#1}}}
\newcommand{\SMVfigdir}{../../../fig/smv/figures}

\definecolor{AQUAMARINE}{rgb}{0.49804,1.00000,0.83137}
\definecolor{ANTIQUE WHITE}{rgb}{0.98039,0.92157,0.84314}
\definecolor{BEIGE}{rgb}{0.96078,0.96078,0.86275}
\definecolor{BLACK}{rgb}{0.00000,0.00000,0.00000}
\definecolor{BLUE}{rgb}{0.00000,0.00000,1.00000}
\definecolor{BLUE VIOLET}{rgb}{0.54118,0.16863,0.88627}
\definecolor{BRICK}{rgb}{0.61176,0.40000,0.12157}
\definecolor{BROWN}{rgb}{0.64706,0.16471,0.16471}
\definecolor{BURNT SIENNA}{rgb}{0.54118,0.21176,0.05882}
\definecolor{BURNT UMBER}{rgb}{0.54118,0.20000,0.14118}
\definecolor{CADET BLUE}{rgb}{0.37255,0.61961,0.62745}
\definecolor{CHOCOLATE}{rgb}{0.82353,0.41176,0.11765}
\definecolor{COBALT}{rgb}{0.23922,0.34902,0.67059}
\definecolor{CORAL}{rgb}{1.00000,0.49804,0.31373}
\definecolor{CYAN}{rgb}{0.00000,1.00000,1.00000}
\definecolor{DIMGRAY }{rgb}{0.41176,0.41176,0.41176}
\definecolor{EMERALD GREEN}{rgb}{0.00000,0.78824,0.34118}
\definecolor{FIREBRICK}{rgb}{0.69804,0.13333,0.13333}
\definecolor{FLESH}{rgb}{1.00000,0.49020,0.25098}
\definecolor{FOREST GREEN}{rgb}{0.13333,0.54510,0.13333}
\definecolor{GOLD }{rgb}{1.00000,0.84314,0.00000}
\definecolor{GOLDENROD}{rgb}{0.85490,0.64706,0.12549}
\definecolor{GRAY}{rgb}{0.50196,0.50196,0.50196}
\definecolor{GREEN}{rgb}{0.00000,1.00000,0.00000}
\definecolor{GREEN YELLOW}{rgb}{0.67843,1.00000,0.18431}
\definecolor{HONEYDEW}{rgb}{0.94118,1.00000,0.94118}
\definecolor{HOT PINK}{rgb}{1.00000,0.41176,0.70588}
\definecolor{INDIAN RED}{rgb}{0.80392,0.36078,0.36078}
\definecolor{INDIGO}{rgb}{0.29412,0.00000,0.50980}
\definecolor{IVORY}{rgb}{1.00000,1.00000,0.94118}
\definecolor{IVORY BLACK}{rgb}{0.16078,0.14118,0.12941}
\definecolor{KELLY GREEN}{rgb}{0.00000,0.50196,0.00000}
\definecolor{KHAKI}{rgb}{0.94118,0.90196,0.54902}
\definecolor{LAVENDER}{rgb}{0.90196,0.90196,0.98039}
\definecolor{LIME GREEN}{rgb}{0.19608,0.80392,0.19608}
\definecolor{MAGENTA}{rgb}{1.00000,0.00000,1.00000}
\definecolor{MAROON}{rgb}{0.50196,0.00000,0.00000}
\definecolor{MELON}{rgb}{0.89020,0.65882,0.41176}
\definecolor{MIDNIGHT BLUE}{rgb}{0.09804,0.09804,0.43922}
\definecolor{MINT}{rgb}{0.74118,0.98824,0.78824}
\definecolor{NAVY}{rgb}{0.00000,0.00000,0.50196}
\definecolor{OLIVE}{rgb}{0.50196,0.50196,0.00000}
\definecolor{OLIVE DRAB}{rgb}{0.41961,0.55686,0.13725}
\definecolor{ORANGE}{rgb}{1.00000,0.50196,0.00000}
\definecolor{ORANGE RED}{rgb}{1.00000,0.27059,0.00000}
\definecolor{ORCHID}{rgb}{0.85490,0.43922,0.83922}
\definecolor{PINK}{rgb}{1.00000,0.75294,0.79608}
\definecolor{POWDER BLUE}{rgb}{0.69020,0.87843,0.90196}
\definecolor{PURPLE}{rgb}{0.50196,0.00000,0.50196}
\definecolor{RASPBERRY}{rgb}{0.52941,0.14902,0.34118}
\definecolor{RED}{rgb}{1.00000,0.00000,0.00000}
\definecolor{ROYAL BLUE}{rgb}{0.25490,0.41176,0.88235}
\definecolor{SALMON}{rgb}{0.98039,0.50196,0.44706}
\definecolor{SANDY BROWN}{rgb}{0.95686,0.64314,0.37647}
\definecolor{SEA GREEN}{rgb}{0.32941,1.00000,0.62353}
\definecolor{SEPIA}{rgb}{0.36863,0.14902,0.07059}
\definecolor{SIENNA}{rgb}{0.62745,0.32157,0.17647}
\definecolor{SILVER}{rgb}{0.75294,0.75294,0.75294}
\definecolor{SKY BLUE}{rgb}{0.52941,0.80784,0.92157}
\definecolor{SLATEBLUE}{rgb}{0.41569,0.35294,0.80392}
\definecolor{SLATE GRAY}{rgb}{0.43922,0.50196,0.56471}
\definecolor{SPRING GREEN}{rgb}{0.00000,1.00000,0.49804}
\definecolor{STEEL BLUE}{rgb}{0.27451,0.50980,0.70588}
\definecolor{TAN}{rgb}{0.82353,0.70588,0.54902}
\definecolor{TEAL}{rgb}{0.00000,0.50196,0.50196}
\definecolor{THISTLE}{rgb}{0.84706,0.74902,0.84706}
\definecolor{TOMATO }{rgb}{1.00000,0.38824,0.27843}
\definecolor{TURQUOISE}{rgb}{0.25098,0.87843,0.81569}
\definecolor{VIOLET}{rgb}{0.93333,0.50980,0.93333}
\definecolor{VIOLET RED}{rgb}{0.81569,0.12549,0.56471}
\definecolor{WHITE}{rgb}{1.00000,1.00000,1.00000}
\definecolor{YELLOW}{rgb}{1.00000,1.00000,0.00000}

\floatstyle{boxed}
\newfloat{notebox}{H}{lon}
\newfloat{warning}{H}{low}

% Set default longtable alignment
\setlength\LTleft{0pt}
\setlength\LTright{0pt}
